\textcolor{\mycolor}{
Given the excess observed in 2016 dataset, the immediate target of the first WP is to establish or exclude signal. Before claiming an observation, careful scrutiny of background contribution is needed. The ongoing effort is focused on validation of Monte Carlo modelling of $t\bar{t}+X$ backgrounds and checking potential contributions from sources with fake leptons, such as events with hadronic jets misidentified as muons or electrons. Decay of four top quarks in the single-lepton mode results in ten jets in the final state. Matrix-element predictions of such high jet multiplicity $t\bar{t}$ events are not possible with existing event generators, therefore the background modelling in this corner of phase space relies on parton shower models that typically have several free parameters fitted to the data. Comparison of $t\bar{t}+X$ predictions based on different parton shower models is one the steps towards understanding of the observed excess.}

\textcolor{\mycolor}{
In case no deviations from the SM will be found and given that the systematic and statistical uncertainties analysis framework will be established, the search results can be reinterpreted using EFT models. \textbf{The ongoing analysis is an extension of the previous results obtained using 2015 dataset~\cite{Sirunyan:2017tep}. My experience in both projects makes me an ideal candidate to carry out this project.} The dataset collected in 2017 will be sufficient to make statistically significant observation of $t\bar{t}t\bar{t}$ SM production. Taking these conditions into account, this WP is considered as low-risk.
}
%The Standard Model measurement in of the \fourtop production cross sections in different channels will be \textbf{published in a refereed journal}.