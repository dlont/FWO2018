In case no deviations from the SM will be found and given that the systematic and statistical uncertainties analysis framework will be established, in principle, the direct searches can be turned into the measurement of the SM \fourtop production cross section with a reasonable effort. In this case, in contrast to the direct searches, probably a wider phase space can be investigated in order to increase the amount of the SM signal \fourtop events. Thus, additional studies of the description of the data by the simulations might be needed in order to ensure reliability of the estimates of the detector effects in a wider phase space. 

\textbf{This analysis is a natural extension of the results obtained using 2015 dataset~\cite{CMS:2016wig}. My experience here makes me an ideal candidate to carry out this project that can be used as a back-up scenario ``if everything else fails''.} As demonstrated in Fig.~\ref{fig:combined_limitvslumi}, even in case of downward statistical fluctuation, the future dataset collected in 2017--2018 will be sufficient to make an observation. Therefore this WP is considered as low-risk.

The Standard Model measurement in of the \fourtop production cross sections in different channels will be \textbf{published in a refereed journal}.