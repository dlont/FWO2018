\textcolor{\mynew}{
The standard model four top production is characterised by moderate transverse momentum of produced tops, therefore decay products can be resolved in the detector. At $\sqrt{s}$=14 TeV, however, a significant fraction of top quarks is expected to be produced with large Lorentz-boost, leading to strongly collimated decay products which end-up in a single jet. 
Moreover, as was mentioned, many BSM models predict heavy states, resulting in a large boost of the decay products. Thus the searches for such states in resolved topology events, as in WP1, are less efficient and reconstruction of boosted top decays becomes crucial for direct searches. 
}
\textcolor{\mynew}{
The top-tagging is currently an active area of research and many novel techniques were proposed over the last years. In particular, the HepTopTagger algorithm~\cite{heptoptagger} demonstrated very good performance in the recent search for pair production of vector-like T quarks in events with similar event signature in CMS~\cite{Khachatryan:2015oba}. Besides that, the CMSTopTagger~\cite{Kaplan:2008ie} employs a different jet deconstruction approach which is more suitable for highly boosted top decays. More algorithms exist including shower-deconstruction tagger and N-subjettiness algorithm, the applicability of which~\cite{CMS:2014fya} still has to be explored. Therefore, in order to enhance sensitivity to BSM \fourtop signal, especially in the zero-leptons decay channel, the investigation of different techniques is foreseen in this project. To my knowledge, besides the discontinued attempt of $W$-tagging at UGent, this can be the \textbf{first significant effort on jet substructure in Belgium}.
}
\textcolor{\mynew}{
This task goes hand-in-hand with optimisation of the $b$-tagging in sub-jet environment. A better performance of the $b$-tagging can give a significant gain in terms of light-flavour background rejection, therefore I plan to investigate novel tools designed in CMS~\cite{Bertolini:2014bba} and explore new promising jet-charge identification technique developed recently in ATLAS~\cite{Aad:2015cua,ATLAS:2015jetcharge}.
}
\textcolor{\mynew}{
This WP is exploratory in nature and assessed as a \textbf{high-risk and high-impact}. This effort is very timely with the LHC instantaneous luminosity increase foreseen for the year 2018 and may have a positive impact for other studies in the collaboration, for example, for 'high-profile' $ttH \rightarrow ttb\bar{b}$ search planned to be performed by the hosting group. The all-hadronic channel may require more significant effort investment. There are diverse challenges inherent to severe not completely understood QCD background. In addition, input from CMS collaborators on performance of b-tagging algorithms in boosted regime is needed. Eventually, the success in this project relies on novel tools. Nevertheless, I believe, I am a perfect candidate for this pioneering effort, since I have a wide knowledge acquired during my doctoral studies in the area of track reconstruction and broad experience in jet and top analyses.
}
\textcolor{\mynew}{
Without doubt, the discovery of a new state will have major impact on the field, but negative result is also worth an effort since very little is known about this extreme corner of phase space. The obtained results will be \textbf{published in a refereed journal}. As previously mentioned, the development of substructure algorithms is of great interest in the field. Therefore, it is expected that new developments will be included in a \textbf{performance paper} by the CMS collaboration.
}
%A more general approach to top tagging, applicable in both boosted and un-boosted topologies, is the Matrix-Element (ME) method. In the ME method, a statistical classifier based on final-state kinematic information and LO matrix elements is constructed to test how likely is that given event originates from signal or background source. This technique was recently demonstrated in $ttH \rightarrow ttbb$ analysis~\cite{MartixElementMethod} and proved to be very effective. In addition to that, the approach can be generalised to decide, for example, whether a given combination of jets and leptons arise from the common top decay. Nevertheless, using theoretical input in the selection procedure, unavoidably introduces model dependence in the measurements. Therefore I propose a possibly novel approach in which this problem can be facilitated by combining ME method with MVA techniques. I propose instead of using weights (which are typically available at LO only) directly in the statistical classifier, to utilise them for seeding multivariate regression model to derive improved weights using available higher-order calculations and/or data.

%The studies for single- and two-leptons channels will largely benefit from the experience developed by the host institute, while in the zero-leptons channel my expertise in QCD jet analysis will be a valuable complement.%As described above, the reconstruction of \fourtop final-state suffers from combinatorial ambiguity in assigning the decay products to theirs parent $t$-quarks. This feature helps to mitigate combinatorial problem, because it suggests natural candidates for decay products originating from the same parton. 

%In the single-lepton channel one can benefit from the constrained kinematics of the event. If the lepton is well isolated and its momentum is reliably measured, the four-momentum of the neutrino from the $W$ decay can be reconstructed as well. Considering a single $t\bar{t}$ pair, four equations: momentum conservation, two constraints due to the $W^\pm$ mass and equality of $t$ and $\bar{t}$ masses, can be used to determine four unknown components of the neutrino momentum vector. This information will help to improve the resolution of the various quantities derived from the momenta of final-state objects. In principle, same technique can be applied in the zero-lepton channel as well, although due to much worse precision of the jet measurement and even larger combinatorial ambiguity, the solution of the constrain equations can be less robust.