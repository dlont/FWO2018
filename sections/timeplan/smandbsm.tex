\textcolor{\mynew}{
Given the first indications of an excess observed in 2016 dataset, the immediate target of the first WP is to establish or exclude the signal of new physics. Before claiming an observation, careful scrutiny of various backgrounds is needed. The ongoing effort is focused on validation of Monte Carlo modelling of $t\bar{t}+X$ backgrounds and checking potential contributions from sources with fake leptons, such as events with hadronic jets misidentified as muons or electrons, etc. 
}
\textcolor{\mynew}{
Decay of four top quarks in the single-lepton mode results in ten jets in the final state. Matrix-element predictions of such high jet multiplicity $t\bar{t}$ events are not possible with existing event generators, therefore, background modelling in this corner of phase space relies on parton shower models that typically have several free parameters fitted to the data. Comparison of $t\bar{t}+X$ predictions based on different parton shower models is one step towards better understanding of the observed excess. Alternatively, data-driven estimate of the $t\bar{t}$ cross section in high jet multiplicity events can be obtained from extrapolation from better understood low multiplicity regions assuming so-called "staircase"~\cite{Ellis:1985vn} and "Poisson"~\cite{Gerwick:2012hq} scaling predicted by QCD. Similar technique was applied in ATLAS~\cite{Aaboud:2017faq}.
}
\textcolor{\mynew}{
Whenever the data sample is fixed and backgrounds are reasonably understood, the measurement of signal cross section is a well-established process. A fit to the multivariate discriminator distribution will be performed in order to estimate the amount of \fourtop events in selected samples. A simultaneous fit to different background- (signal-) enhanced event categories can be performed to constrain the background and determine the number of signal events. I have a good knowledge of various fitting approaches, including multidimensional likelihood fits to the distributions with low counting statistics to effectively work on this task.
}
\textcolor{\mycolor}{
In both BSM searches and SM measurement the experimental and theoretical sources of systematic uncertainty have to be taken into account as both can affect the shape and normalisation of the measured distributions. Examples of theoretical uncertainty include the ones due to the variation of renormalisation and factorisation scales in perturbative calculations or the variation of the matching scale in the parton-shower-matching procedure. The experimental uncertainties arise, for example, due to finite accuracy of jet energy calibration, b- and top-identification procedures, luminosity etc. These uncertainties can be included as nuisance parameters into the above-mentioned likelihood fits.
}
\textcolor{\mycolor}{
The dataset collected in 2017 will be sufficient to make statistically significant observation of $t\bar{t}t\bar{t}$ SM production. However, input from CMS collaborators working on estimation of trigger efficiencies, calibrations and determination of scale factors will be needed. Due to many similarities of two channels, the analysis of single-lepton and opposite-sign dilepton events can be performed in parallel together with PhD student, which I am currently co-supervising. Individual channels will differ in the details of systematic studies, mainly because of the variations in the background strength and sources composition. 
}

\textcolor{\mycolor}{
The ongoing effort is an extension of the previous results obtained using 2015 dataset~\cite{Sirunyan:2017tep}. First results with 2016 data, outlined in the previous sections, \textbf{demonstrate feasibility} of this WP. Moreover, a backup scenario for background estimation is envisaged. \textbf{My experience in ongoing and previous projects makes me an ideal candidate to carry out described tasks.}
The success in WP1 partially depends on external input, however the analysis is based on well-established tools and follows standard strategy adopted in the field. Taking these conditions into account, this WP is considered as \textbf{medium-risk.} \textbf{An observation of BSM process will have significant impact on the field, but even in case of null results the studies of SM \tttt process are novel enough to be published in a peer-reviewed journal.}
}