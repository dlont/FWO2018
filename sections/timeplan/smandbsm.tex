
A search for and first observation of SM $t\bar{t}t\bar{t}$ production in the single-lepton channel is the first goal of this proposal. 
Whenever the data sample is fixed and backgrounds are reasonably understood, the measurement of signal cross section is a well-established process. For the low-level calibrations such as (in)efficiencies and scale factors necessary to get agreement between data and simulation, the intention is to rely on the software framework used in the VUB group, which is shared between many CMS analyses in top physics and includes all necessary corrections. However, it will be necessary to make sure that the estimation of trigger efficiencies, calibrations and determination of scale factors is sufficient before this effort can start.

Given the first indications of an excess observed in 2016 dataset, the immediate target of the first WP is to establish or exclude the signal using the 2017 and 2018 LHC datasets.  
Before claiming an observation, careful scrutiny of various backgrounds is needed, including detailed study of Monte Carlo modelling of $t\bar{t}+X$ backgrounds and checking potential contributions from sources with fake leptons, such as events with hadronic jets misidentified as muons or electrons, etc. 

Decay of four top quarks in the single-lepton mode results in ten jets in the final state. Matrix-element predictions of such high jet multiplicity $t\bar{t}$ events are not possible with existing event generators, therefore, background modelling in this corner of phase space relies on parton shower models that typically have several free parameters fitted to the data. Comparison of $t\bar{t}+X$ predictions based on different parton shower models is one step towards better understanding of the observed excess. Alternatively, data-driven estimate of the $t\bar{t}$ cross section in high jet multiplicity events can be obtained from extrapolation from better understood low multiplicity regions assuming so-called "staircase"~\cite{Ellis:1985vn} and "Poisson"~\cite{Gerwick:2012hq} scaling predicted by QCD. Similar technique was applied in ATLAS~\cite{Aaboud:2017faq}.

To measure the $t\bar{t}t\bar{t}$ signal, a fit to multivariate discriminator distribution (obtained with machine learning) will be performed in order to estimate the amount of \fourtop events in the data. A simultaneous fit to different background- (signal-) enhanced event categories will be performed to constrain the background and determine the number of signal events. I have a good knowledge of various fitting approaches, including multidimensional likelihood fits to the distributions with low counting statistics to effectively work on this task.

The experimental and theoretical sources of systematic uncertainty for such a method have to be taken into account as both can affect the shape and normalisation of the measured distributions. These uncertainties will be included as nuisance parameters into the above-mentioned likelihood fits.
%Examples of theoretical uncertainty include the ones due to the variation of renormalisation and factorisation scales in perturbative calculations or the variation of the matching scale in the parton-shower-matching procedure. The experimental uncertainties arise, for example, due to finite accuracy of jet energy calibration, b- and top-identification procedures, luminosity etc. 
The most important systematic uncertainties in a \fourtop analysis are from the simulation of $t\bar{t}$ production, and the b-identification. For the former I have unique experience due to my role as MC contact, and the latter the experts are present at the VUB. 

Assuming the currently known SM cross section for $t\bar{t}t\bar{t}$ production, the complete LHC Run 2 dataset collected in 2018 should be sufficient to make a statistically significant observation of $t\bar{t}t\bar{t}$ SM production, which would be an important result from CMS. However, considering the first results outlined in the previous sections, there are hints that $t\bar{t}t\bar{t}$ could be observed earlier, which is what WP2 focuses on. 

The analysis of single-lepton events can be performed in parallel with a PhD student from University of California Riverside who focuses on the opposite-sign dilepton final state and who I am currently co-supervising. 
%Individual channels will differ in the details of systematic studies, mainly because of the variations in the background strength and sources composition. 

The ongoing effort is an extension of the previous results obtained using 2015 dataset~\cite{Sirunyan:2017tep}. First results with 2016 data, outlined in the previous sections, \textbf{demonstrate feasibility} of this WP. Moreover, a backup scenario for background estimation is envisaged. \textbf{My experience in ongoing and previous projects makes me an ideal candidate to carry out described tasks.}
The success in WP1 partially depends on external input, however the analysis is based on well-established tools and follows standard strategy adopted in the field. Taking these conditions into account, this WP is considered as \textbf{medium-risk.} \textbf{An observation of the $t\bar{t}t\bar{t}$ process will have significant impact on the field as the first observation of a BSM-sensitive SM \tttt process is  novel enough to be published in a high-impact peer-reviewed journal.}

