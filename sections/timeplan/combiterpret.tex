\textcolor{\mynew}{
The aim of this WP is to perform statistical combination the results obtained in different channels and interpret searches performed in WP1 and WP2 using phenomenological models. As was shown by the applicant in~\cite{Sirunyan:2017tep}, combination of SM \tttt searches in different channels can substantially improve the sensitivity of the search, due to complementarity of information provided by different channels. Although, the improvement in the referenced publication, is attributed mainly to increase of the signal efficiency due to summing contributions from different decay modes, the synergy of independent channels will become more prominent, when statistical component of the uncertainty can be reduced using larger datasets accumulated in 2016--2018. For example, rates of rare backgrounds, such as $t\bar{t}Z$ and $t\bar{t}W$, contributing in both single-lepton and dilepton channels can be better constrained in dedicated control regions of dilepton analysis and applied in single-lepton search. On the other hand, dilepton analysis can profit from larger branching fraction of $t\bar{t} \rightarrow l+\mathrm{jets}$ to better constrain jet multiplicity spectrum, the modelling of which was among dominant sources systematic uncertainty, so far. Combination will be performed using statistics tools, that are widely used in ATLAS and CMS experiments. The applicant has all necessary skills to effectively accomplish this task.
}
\textcolor{\mynew}{
A novel goal of this WP is a bottom-up interpretation of the results of the searches using four top EFT model~\cite{DegrandeEFTthesis}. In case, when no BSM states will be observed, the limits on coupling of effective operators will provide constraints on potential physics scenario beyond the standard model. In contrast to phenomenological interpretation using specific BSM models, such as e.g.~\cite{Sirunyan:2017uyt}, where very restrictive assumptions on model parameters had to be made ($\tan \beta=$1\footnote{In this analysis, the so-called 2HDM model was considered, where $\tan \beta$ is a ratio of vacuum expectation values of $H_u$ and $H_v$ fields.}), EFT analysis helps to draw more universal conclusions about potential new physics. Thus, such results will extend the scope and impact of existing interpretations of four top production data. 
}
\textcolor{\mynew}{
An initial attempt of EFT interpretation of 2015 results, described above, \textbf{demonstrates the potential of new data and my capability to successfully perform this sub-project}. The outlined analysis, is limited by assumption that detector acceptance of \tttt final state resulting from effective interaction of top quarks is the same as for SM \tttt production, which is not the case if the EFT cut-off scale is high, as mentioned in WP2. In order avoid potential ambiguities, the impact on experimental sensitivity of  differences in two production mechanisms will be taken into account in detailed CMS simulations of EFT processes. There are only 5 dim-6 four-fermion operators contributing to production of four top quarks in proton-proton collisions. However, model parameters scan quickly becomes a formidable task, as soon as full detector simulations are employed. Nevertheless, polynomial dependence of EFT predictions on model parameters allows to interpolate calculations using only a small set reference computation, thus, making this goal achievable. \textbf{My experience with the MC generation in CMS PdmV group is relevant to ensure timely production of necessary MC simulations.} This analysis will complement ongoing studies in UGent, where other rare $t\bar{t}W$ and $t\bar{t}Z$ processes are analysed within EFT framework. All these studies fit nicely with more ambitious effort of SM EFT analysis of all existing $t\bar{t}$ data, which is also carried out at VUB and XXX. 
}
\textcolor{\mynew}{
The two sub-projects detailed in this WP are assessed as low and medium risk, respectively. The first sub-project is a new incarnation of the previous studies, extended to all-hadronic channel. While the second sub-project is more innovative, but is build on my expertise developed in the past year. The work on these projects can be started as soon as first results with new data appear. The outcome of these all will be published \textbf{together with the experimental results} outlined above or in a \textbf{separate publication}.
}
%The modified frequentist $\mathrm{CL_s}$ approach~\cite{Junk:1999kv,Read:2002hq} can be used to assess the statistical significance of the signal in case of observation or to establish the limits on the mass and production cross section, as well as to combine results obtained in different channels. \textbf{This WP will build on the expertise of other researchers in the institution and strong collaboration with the authors of respective theoretical predictions and thus has potential risks.}  A bottom-up approach would be the interpretation of the data using the EFT calculations. together with cooperation with the phenomenology groups in Brussels is the perfect fit to complete this task in an effective manner.
%
%\begin{itemize}
%\item EFT
%\begin{itemize}
%\item dim-6 four fermion
%\item limitation of Bsc analysis is the assumption of similarity of acceptance for EFT and SM 
%\item proof of feasibility 
%\item analytic parametrisation
%\item full or fast mc simulations
%\end{itemize}
%\item +Combination of sl, dl and all hadronic
%\item +higgs combine tool
%\item +experience in combination of 2015
%\end{itemize}
%
