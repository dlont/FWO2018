\begin{center}
\textbf{\textit{{\Large Outline how a two way transfer of knowledge will occur between you and the host institution, in view of their and your future development and past experience.}}}\\
\end{center}
There are two main aspects of the proposed research that are new to me and will have a major impact on the capacity to increase future career perspective. 

1.	This project allows to gain the expertise in the field of proton-proton collisions at the energy frontier, which is an ideal complement to my current experience in electron-proton and proton-antiproton high-energy reactions. So far, I specialised in precision measurements, while this study will be dominantly  focused on direct searches for new physics phenomena and help me develop into complete physicist. Thus, this research will \textbf{significantly extend my field of expertise and improve my skills in scientific project management}. The host institution provides very strong support in attending regular collaboration meetings at CERN (Switzerland) as well as conferences and workshops in Belgium and world-wide. The institution exerts a significant effort to maintain healthy scientific environment and encourages continuous flow of ideas between local groups working in different areas of elementary particles and quantum fields. The institutional seminars provide a unique opportunity to learn from leading experts working in the fields as diverse as neutrino cosmology and astronomy, LHC physics, dark matter and quantum field theory.

2.	I was recently selected to the, so-called Level-3, coordinator in the Monte Carlo (MC) generation group in CMS. Among others, the main duties at this post include implementation of the collaboration policies in the area of experiment-wide MC simulations effort as well as chairing of the regular meetings. Thus, this commitment will also \textbf{advance my knowledge in MC generators}. Moreover, this mandate will be perfect to \textbf{develop strong leadership skills in a rapidly changing environment} on the forefront of particle physics and will be absolutely crucial for my future career development.

As was mentioned, I have a strong expertise in the precision measurements of high-energy reactions, phenomenological data analysis and particle tracking, therefore my knowledge is in perfect accord with current host group endeavour in the top-quark sector. I will share my expertise by initiating a pioneering project dedicated to the development and optimisation of the boosted top quark tagging algorithms. These results will be discussed on internal and collaboration meetings and will lead to multiple publications in a refereed journal.

\begin{center}
\textbf{\textit{{\Large Outline the quality of the supervision and the hosting arrangements.}}}\\
\end{center}
The VUB is a strongly research-oriented university that was ranked \textbf{5th in Belgium} by the QS World University Ranking in 2015. Only the IIHE published \textbf{more than 140 papers} in 2015. In general, the laboratory encompasses about 50 senior researchers and postdocs as well as about 40 students from VUB and ULB universities and is tightly linked with respective physics departments. The institute has well established scientific networks with other Belgian and international universities, in particular UGent, UAntwerpen and UCLouvain as a member of the CMS collaboration. The institute is \textbf{well recognised for its experience in management of large multinational scientific groups}, for example Prof. D'Hondt is the head of the CMS collaboration board, and at least one faculty member is typically selected to the CMS physics group coordinators panel. The senior researchers and postdocs are actively engaged in training activities, in order to share their expertise and smoothly integrate newcomers into the group's research effort. For example, Prof. Blekman has recently given a master-class oriented to doctoral students and young postdocs at the ``Data Analysis School''. Young researchers are strongly encouraged to participate in topical conferences in order to rapidly get up to date in the field. As such, the IIHE has an excellent track in training postdoctoral and fellow researchers. One particular demonstration of this is a large fraction of postdocs in the institute, selected in CMS for coordinating roles, while, in general, only about 10\% of post-doctorates hold such positions in the experiment.

 Besides outstanding training opportunities, the IIHE has as well developed infrastructure. In particular, the laboratory hosts the \textbf{unique in Flanders Tier-2 grid-computing cluster} essential for the particle physics research.

Overall, this demonstrates exceptional quality of the supervision and host-institution infrastructure as well as that comprehensive set measures are taken in order to integrate new members within the lab and, as such, that the VUB will be a perfect place for the project realisation and will help me to accomplish as a scientist.

\begin{center}
\textbf{\textit{{\Large Explain on how you will establish a solid communication and public engagement strategy of the research.}}}\\
\end{center}
As was mentioned, the findings of this project will be disseminated to the scientific community via \textbf{at least three publications in the peer-reviewed journals}. Furthermore, in order to ensure wider availability, the results will be also published at the \texttt{arXiv} preprint service. Besides that, the results will be delivered in the form of \textbf{presentations or posters at the major topical conferences}. In addition, the general policy of the CMS collaboration requires that all stages of the research must be detailed in a dedicated report and distributed among the collaboration members, prior they become public.

I am very enthusiastic about the opportunity to communicate scientific results to general audience and, especially, as a professional researcher, to encourage young individuals to study science. In this respect, several activities targeted at high-school and university students are planned.\\
\textbf{Academic institution activities.}\\
The VUB regularly organises infodays delivered in English for international Master students, in order to get acquainted with professors, visit laboratories and get to know to students life, therefore the participation in such or similar events is planned during the mandate. Besides local activities I admire the opportunity to use this mobility program to communicate the research and Marie Sk\l{}odowska-Curie fellowship approach to my homeland. In accord with the general European strategy for strengthening the relationships between the EU and third countries, I plan to give a presentation at the physics faculty of the Taras Shevchenko University of Kyiv and Bogolyubov institute for theoretical physics\footnote{At the time of writing a preliminary agreement with responsible representatives of the institutions were obtained. The presentation can be held upon arrangement.} during my visit. The performed research will be described and an informal meeting with graduate students dedicated to explaining the advantages of the Marie Sk\l{}odowska-Curie and FWO international mobility programs can be hold. As the former institution is the leading university and the later is an authority in theoretical physics in Ukraine, this will effectively increase the awareness among high-profile students and enhance the scientific exchange between Ukraine and the EU.
\newline
\textbf{School visits.}\\
In order to popularise science among students and to help teachers at secondary schools to tune physics curriculum to fit with the modern scientific studies, I plan to give few classes on particle physics research and hold several Q\&A sessions at Brussels international schools as well as secondary schools in Ukraine\footnote{The support from one natural-sciences-oriented and two general-profile schools in Ukraine was obtained already.}. 