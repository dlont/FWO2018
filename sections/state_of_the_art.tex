It is generally accepted that there are four fundamental forces that govern the evolution of the Universe. It is believed that the successful resolution of the so called hierarchy problem of particle physics, namely, why the gravitational interaction is many orders of magnitude weaker than other forces, is an essential ingredient to the resolution of a wider problem, the construction of self-consistent description of quantum gravity and a unified theory of all interactions.

The Standard Model (SM) of particle physics is a quantum field theory describing strong, weak and electromagnetic interactions of elementary matter constituents. Since its inception in the 1960s, the SM was placed under high scrutiny in a vast range of experiments. The predictions of the SM were tested at the scales down to $10^{-19}$m and no significant deviations were observed so far~\cite{Agashe:2014kda}. For example, the perturbative predictions of quantum chromodynamics (QCD), the theory of strong interaction of quarks and gluons, provide a very good agreement in the description of the \textbf{precise measurement of the inclusive-jet cross sections in $ep$ collisions at HERA} (Hadron-Electron Ring Accelerator)~\cite{Abramowicz:2012jz}, to which I made the main contribution. 

The observation~\cite{Aad:2012tfa,Chatrchyan:2012xdj} of the quantum of the Higgs field, predicted by the BEH (Brout--Englert--Higgs) mechanism of electroweak symmetry breaking, in 2012 marks the completion of experimental verification of the particle content of the SM. Nevertheless, the SM cannot be considered as ultimate theory because it has several limitations. In particular, the SM does not explain a priory why the scale of electroweak symmetry breaking is so much different from the Plank scale, at which all fundamental interactions are supposed to unify. In other word, the SM doesn't answer why gravitational force is much weaker than other interactions. Moreover, the huge difference of about seventeen orders of magnitude between the two scales is considered as very unnatural and suggests to look for mechanisms beyond the SM (BSM) capable to generate such a ``hierarchy'' of scales. In addition, there is neither explanation for observed matter-antimatter asymmetry of the Universe nor for the origin of dark matter and energy.

Various models extending the content of the SM were proposed over the past decades to provide a natural (without excessive fine-tuning of the model parameters) solution to the hierarchy problem. Suggested models can be approximately grouped into several classes including supersymmetric (SUSY) extensions of the SM, models with extra dimensions, composite Higgs and the so-called little-Higgs models. \textbf{Most of the proposed solutions predict new particles with enhanced coupling to the heaviest SM particle --- the top quark}\footnote{In what follows, (anti-)top quarks are generically called top and denoted by $t$, unless explicitly stated.}. Among those are models with vector-like quarks~\cite{Aguilar-Saavedra:2013qpa}, models that predict additional massive color-singlet spin-1 $Z'$~\cite{Langacker:2008yv} or composite Higgs~\cite{Agashe:2004rs} boson, models with the so-called Kaluza-Klein (KK) states~\cite{Agashe:2006hk,Davoudiasl:1999jd}, as well as those with strongly interacting scalar fields --- sgluons~\cite{Plehn:2008ae,GoncalvesNetto:2012nt}. Although a wealth of theoretically viable models have been proposed, only an empirical evidence can establish their verity, therefore the \textbf{search for BSM processes is one of the main objectives of the experiments at the Large Hadron Collider (LHC)}, particularly in the so called RUN2 with increased collision energy and therefore larger cross section of hypothetical BSM processes. An observation of such processes will be a \textbf{major breakthrough in the field} after the Higgs boson discovery.

Two general-purpose experiments at the LHC, ATLAS and CMS, have extensive physics programs dedicated to BSM searches. The state-of-the-art analyses, focused on final states with a top quark signature, utilised $\sqrt{s}=$ 13 TeV data to establish the upper limit on the SM four top quarks production cross section~\cite{ATLAS:2016gqb,CMS:2016wig,Beck:2016hyi}. These new data can be re-interpreted within the BSM context as, for example, in~\cite{Beck:2015cga}, where %narrow leptophobic $Z'$ and gluon KK resonances decaying to $t\bar{t}$ pairs with a mass below 2.4 and 2.8 TeV, respectively, were excluded at 95\% confidence level (CL) by CMS~\cite{Khachatryan:2015sma}; vector-like charge $2/3e$ T quarks with lower mass limit between 720 and 920 GeV were excluded exploiting events with $ttHH$ and $ttWW$ in the final state~\cite{Khachatryan:2015oba};
the $\sqrt{s}=$ 8 TeV results were used to derive the lower limit on the mass of scalar gluons up to 750 GeV assuming typical sgluon-top coupling values. The hosting group and recently the author accumulated the unique expertise in this area. BSM models with enhanced coupling to the top quark are constrained by SUSY searches looking for gluino\footnote{Gluino is a fermionic SUSY partner of the gluon.}-mediated stop\footnote{Stop is a scalar SUSY partner of the top quark.} production~\cite{CMS:2014dpa,Khachatryan:2016kod}. However, in contrast to the study to be pursued in this project, these searches impose very strict requirements on the missing momentum, \misspt, due to undetected weakly interacting particles\footnote{Popular SUSY models, including the MSSM, endowed with the so-called R-symmetry, contain lightest neutral weakly-interacting SUSY particle (LSP) in the mass spectrum. Thus, heavier SUSY partners decay to SM particles and the LSP resulting in significant missing momentum as an experimental signature.}.

The mentioned analyses exploit properties of the top-quark decays to enrich data samples with hypothetical signal events. Top quarks decay almost exclusively via $t \rightarrow Wb$ channel, therefore initial $t$-selection is typically based on identifying secondary vertices from $b$-decays and leptons or jets from $W$-bosons. Additionally, if the $W$-boson decays via $W\rightarrow l\tilde{\nu}$, a small amount of \misspt is also present in the event and attributed to neutrinos in the SM. 

Decays of heavy objects, like  aforementioned BSM states, result in large Lorentz-boost of daughter particles, therefore the decay products of the top quark, that originates from such a state, typically overlap in the detector. In order to overcome this problem and achieve high signal selection efficiency and purity\footnote{Efficiency determines the relative amount of signal evens surviving selection procedure, while purity is a measure of background contribution.}, i.e. optimise statistical significance\footnote{For large number of event counts the statistical significance in counting experiments is approximately equal to the ratio $S=N_{\mathrm{sig}}/\sqrt{N_{\mathrm{sig}}+N_{\mathrm{bg}}}$, where $N_{\mathrm{sig}}\left(N_{\mathrm{bg}}\right)$, denotes an estimate of the number of signal (background) events.} of hypothetical signal, dedicated reconstruction procedures are necessary. Above-mentioned studies utilised multivariate analysis (MVA) methods and I propose to also use jet-substructure techniques to tackle this problem. Currently, development and optimisation of reconstruction and identification algorithms for boosted objects is an extremely active area of research in both experimental and theoretical communities.

As mentioned, no deviations from predictions of SM were observed so far. Nevertheless, the 2017--2018 $\sqrt{s}=13$ TeV and planned 14 TeV LHC runs have high discovery potential. As demonstrated by the author (see Fig.~\ref{fig:combined_limitvslumi}), the limit on the SM $t\bar{t}t\bar{t}$ can be significantly improved and possibly observation of the SM signal can be made, provided the dataset that are expected in coming years. 
Given more than thirty-fold increase of the integrated luminosity of the $\sqrt{s}=13$ dataset in 2016--2017, the LHC may provide a decisive answer to the naturalness problem during the time covered by this proposal. With increased amount of data and potentially $pp$ collision energy, larger mass range and more complex event signatures become accessible at the LHC. To efficiently explore these new signatures, novel analyses strategies will be necessary. Such novel analysis is the subject of this proposal.

I believe that investigation of processes with multi-top-quark signature will be an important milestone in coming years and one of the main paths to new discoveries at the energy frontier. Such a discovery would likely be the start of a new revolution in fundamental physics. 