From experimental point of view, the process cross section can be determined applying generic formula:
\begin{equation}
\sigma = \left(N_{\mathrm{sig}}-N_{\mathrm{bg}}\right)\cdot \mathcal{A} \cdot \mathcal{L}^{-1},
\label{eq:csdef}
\end{equation}
where $N_{\mathrm{sig}}\left(N_{\mathrm{bg}}\right)$, denotes an estimate of the number of signal (background) events, while $\mathcal{L}$ and $\mathcal{A}$ represent an integrated luminosity and a correction factor taking detector and possibly other effects into account, respectively. 

As follows from the Eq.~\eqref{eq:csdef}, to measure the process cross section, typically several steps have to be performed, i.e. the number of signal and background events as well as the integrated luminosity have to be determined for a given data sample; the detector effects attributed to e.g. inefficiencies, finite resolution, etc. also have to be taken into account. On the other hand, in the direct searchers for new phenomena, typically the regions of phase space are investigated in which the signal production cross section is enhanced and exceeds the background significantly. In general, in this procedure certain assumptions about e.g. modelling of the detector response or the shape of the background spectrum are typically made. The sensitivity of the result to variations of different assumptions is called systematic uncertainty and is one of the crucial components of the analysis that often requires elaborate studies. Besides that, the precision of the measurements with low event count rate as e.g. \fourtop production, is usually limited by stochastic effects\footnote{Typically attributed to statistical uncertainty on the number of signal events.}, which also have to be properly taken into account. To achieve all mentioned tasks, various techniques, employing simulations of relevant processes, data-driven analysis methods and statistical means, exist. The proposed research follows closely well established paradigm in the field. The foreseen steps and intermediate goals of this study are detailed below.

Depending on the principal task, i.e. measurement of the SM cross section or direct search for BSM signal, two corresponding sub-projects can be identified in this proposal. In general, any experimental analysis aims at best measurement precision, however, in case of searches for new phenomena, the maximal statistical significance of the signal is required in addition.
% The $b$- and $t$-reconstruction and tagging algorithms, to be described below, are being developed in order to minimise background contamination, $N_{\mathrm{bg}}$, and to reconstruct maximum possible number of signal events, $N_{\mathrm{sig}}$ in case of direct searches, and to be maximally robust against detector effects in case of cross section measurement. As follows from Eq.~\ref{eq:csdef}, 
%
Two different strategies will be pursued in the respective sub-projects, however both can be performed using comparable tools. Furthermore, it is natural to split the sub-projects further according to the $t\bar{t}t\bar{t}$ decay channel to be considered, corresponding to zero-, single- and two-lepton final states, respectively. As all three have different experimental signatures, development and optimisation of individual reconstruction algorithms, selection of appropriate trigger chains, background studies as well as determination of statistical and systematic effects will be different in three sub-projects. However, they have a common core related to the reconstruction and identification of $b$-quark jets and reconstruction of hadronic top decays.

The single-lepton channel, previously studied in~\cite{CMS:2016wig,Beck:2016hyi} and~\cite{Khachatryan:2014sca} is considered the most straight-forward analysis in this project. It has the largest branching fraction and expected to have moderate background level. Moreover, constrained-kinematics reconstruction, extensively employed in CMS in top-pair production analyses, is directly applicable in this channel and will have significant effect on top quark identification efficiency. This sub-project will largely benefit from extensive experience of the host institution in this analysis and will use state-of-the-art dedicated tools developed there. Thus, it will be natural to start analysing 2016--2017 data in this channel. In general, this sub-project is considered as a low risk and the \textbf{direct search in this channel represents the minimal goal of the proposal}. The results of this study using 2016--2017 data is already \textbf{novel enough to be published} in a peer-reviewed journal. 

To complete the analysis and draw the conclusions about the impact of new searches, the data can be interpreted within the framework of existing BSM models. In this case the input from the authors of respective predictions is required, and thus the success of this task depends on the collaboration with other researchers. In addition, a detailed phenomenological analysis may be challenging, therefore experimental groups often resort to the so-called simplified models in which very restrictive assumptions about the particle content and/or model parameters are made. For example, the simplest scenario to consider would be assuming hypothetical state of fixed mass and production cross-section that has exclusive coupling to four top quarks. Well established tools will be used to assess the statistical significance of observed signal, if any, or to derive limits on the production cross section and mass, otherwise.

An addition to the two mentioned approaches, would be to interpret the data within the so-called effective-field theory (EFT) approach~\cite{Lillie:2007hd} in which the impact of BSM physics is parametrised by higher-dimensional operators constructed from the products of SM fields. In my opinion, this is the optimal approach, because these results, in contrast to simplified models, can, at least in principle, be reinterpreted within any given BSM model. The EFT-analysis of the measurements is a novel approach to the interpretation of the $t\bar{t}t\bar{t}$ data and may attract larger attention to the obtained results. The outcome of this analysis can be presented as a \textbf{separate publication} or together with the results of the searches. 

Determination of the $t\bar{t}t\bar{t}$ production cross section in other channels and corresponding searches are a logical extension of the described analysis. The two-lepton channel can provide a valuable contribution with a moderate effort, while the analysis in the zero-lepton channel may present significant challenge due to major difference in the final-state. In this case \textbf{my experience in jet analyses will be especially suitable for developing dedicated tools for hadronic top decay} reconstruction.

In order to reach the ultimate sensitivity in the BSM searches, the next logical step, contingent with these sub-tasks, is a combination and interpretation of the results obtained in various channels. They will perfectly fit together with the analysis of the single-lepton channel, but can be issued together with the combination in a \textbf{separate publication}.

To fulfill the ambitious plans outlined above, the research will be delivered in three work-packages which are elucidated in the following.
% baseline selection and event reconstruction
% interpretation of the results within specific and EFT models. Cite Chen:2014ewl